%----------------------------------------------------------------------------------------
%	PACKAGES AND THEMES
%----------------------------------------------------------------------------------------
\documentclass[aspectratio=169,xcolor=dvipsnames]{beamer}
\usetheme{Simple}

\usepackage{hyperref}
\hypersetup{
    colorlinks,
    linkcolor=black, % Overview
    urlcolor=blue    % \href#2
}

\usepackage{graphicx} % Allows including images

%----------------------------------------------------------------------------------------
%	Macros
%----------------------------------------------------------------------------------------
\newcommand{\bi}{\begin{itemize}}
\newcommand{\ei}{\end{itemize}}
\newcommand{\param}[1]{{$\langle#1\rangle$}}
\def\figs{figs2}
%----------------------------------------------------------------------------------------
%	TITLE PAGE
%----------------------------------------------------------------------------------------

% The title
\title[short title]{FAIR Data}
\author{\includegraphics[scale=0.5]{ildg-logo}\\Working Groups}
\institute{Hands-on Workshop}
\date{June 14, 2023 } % Date, can be changed to a custom date


%----------------------------------------------------------------------------------------
%	PRESENTATION SLIDES
%----------------------------------------------------------------------------------------
\begin{document}

\begin{frame}
    % Print the title page as the first slide
    \titlepage

\end{frame}

\begin{frame}{Overview}
    % Throughout your presentation, if you choose to use \section{} and \subsection{} commands, these will automatically be printed on this slide as an overview of your presentation
 %   \tableofcontents

\begin{itemize}
    \item Background to storing data
     \item Introducing FAIR
    \item Lattice data goes FAIR
\end{itemize}
 
\end{frame}

%----------------------------------------------------------------------------------------
\section{Background}

\begin{frame}{Introducing databases}

\begin{itemize}
    \item In the 1960s companies such as IBM started to develop databases to store 
           information in a systematic way
\item Sabre (travel reservation system) was the first commercial large scale database
    \item After some experimentation,  relational database were developed where the information is stored in tables and a query language called SQL was used to extract information from the tables.

    \item There is a schema that defines the column names and data types in the table. 
\end{itemize}

\begin{table}
\begin{tabular}{ c c c}
  FarmID & Name              & Acres \\  \hline
   1     & Black Hallow Farm & 500  \\
   2     & Robinwood Farm    & 4000 \\
\end{tabular}
\caption{Farm table}
\end{table}


\end{frame}

%%%%%%%%%%%%%%%%%%%%%%%%%%%%%%%%%%%%%%%%%%%%%%%%%%%%%%%%%%%%%%%

\begin{frame}{The evolution of databases}

From the 1970s to the late 1990s relational databases were dominant. 

\vspace{1cm}
However
\begin{itemize}
    \item Relational databases didn't map well to object oriented languages such as c++, so object orientated databases were developed (but not widely used.)
\item The development of the original web motivated researchers to develop a \textbf{semantic web}. There is a lot of information stored in a web page.  Tim Berners-Lee wanted it be possible to use this information in a systematic way. The semantic web didn't take off, but one of the technologies developed was XML.

\item The storage of large quantities of social media posts and images motivated researchers to develop databases which were not based on tables.
Examples are graph and document databases. Generically these are known as \textbf{NoSQL databases}.

    
\end{itemize}

\end{frame}

\begin{frame}{What is the dream?} 

From \url{https://www.physics.nat.fau.eu/2021/07/05/fairmat-lifting-the-treasure-trove-of-materials-data/}
Using optoelectronics as an example, the goal is to discover and investigate highly efficient, low-cost, and nontoxic semiconductors with optimal properties for devices and systems for renewable energy generation and conversion.

\vspace{0.5 cm}

\begin{itemize}

\item If scientific data from many different sources (experiments or simulations)  can be searched, then this could speed up scientific discovery


\item Data from the different sources need to be combined.

\item The data needs to be properly described.

\item The data needs to be readable by both humans and \textbf{machines}.

\item Ideally the data needs to be persistent. 
\end{itemize}

Between 2006 and 2016 Google hosted google code project. Archive still exists.

\end{frame}

%================================================
\section{Lattice Data goes FAIR}
%================================================
\begin{frame}{FAIR Principles} 

  \hspace*{7em}
  \parbox{40mm}{
  \alert{F}indable\\
  \hspace*{3em} \alert{A}ccessible \\
  \hspace*{6em} \alert{I}nteroperable \\
  \hspace*{9em} \alert{R}eusable
  }
  \hfill
  \parbox{26mm}{\footnotesize
    \href{https://www.force11.orf}{force11.org}\\[-1mm]
    \hspace*{8mm}{\footnotesize $\vdots$}\\[-1mm]
    \href{https://www.nature.com/articles/sdata201618}{Wilkinson~2016}\\[-1mm]
    \hspace*{8mm}{\footnotesize $\vdots$}\\[-1mm]
    \href{https://www.go-fair.org/fair-principles}{go-fair.org}
  }

  \vspace*{5mm}
  \begin{itemize}
    \item It is becoming a mandatory \alert{requirement} by funding agencies \\
      {\small ``The [European] Commission will work with global policy and research
        partners to foster cooperation and to create a level playing field in
        scientific data sharing and data-driven science.''\\
      \hfill\href{https://eur-lex.europa.eu/legal-content/en/TXT/?uri=CELEX:52016DC0178}{EU Commission, COM(2016)178}
      }
    \item provides \alert{guiding principles}, not an implementation
    \item conceptually refers to three types of entities: %%% FAIR-go
      \begin{itemize}
        \item data = any digital object
        \item metadata (MD) = information about digital object
        \item infrastructure
      \end{itemize}
    \item requires machine actionable (meta)data
  \end{itemize}
\end{frame}

%------------------------------------------------
\begin{frame}{What does ``findable'' mean?}
  %% list requirements from Go Fair: https://www.go-fair.org/fair-principles/ [www.go-fair.org]) xxx}
  \begin{alertblock}{Findable}
    \begin{itemize}
      \item[F1] globally unique and persistent ID assigned to (M)D
      \item[F2] data described with rich MD %%% (see R1) % DP: drop
      \item[F3] MD includes data ID of data
      \item[F4] (M)D registered or indexed in a searchable resource
    \end{itemize}
  \end{alertblock}

  Metadata includes information on
  \begin{itemize}
  \item content (general and domain-specific vocabulary)
  \item provenance (who, when, where, how?)
  \item access (format, path, license, \ldots)
  \item \ldots
  \end{itemize}

\end{frame}
%------------------------------------------------
\newcommand{\access}[1]{
\begin{frame}{How does ILDG address ``findable''?}
  \begin{columns}[c] 
    \column{0.5\textwidth}
    Metadata
    \begin{itemize}
    \item follows a well-defined and rich schema
    \item stored \alert{separately} from data (big)
    \item searchable in central catalog of \alert{each} RG (regional grid)
    \end{itemize}
    
    \column{0.5\textwidth}
    \hspace*{-3mm}
    \includegraphics[height=35mm]{\figs/#1}
  \end{columns}
\end{frame}
}
\access{ildg-access1schema}
\access{ildg-access2}
%------------------------------------------------
\begin{frame}{Unique identifiers}

    \begin{columns}[c] 
      \column{0.70\textwidth}
      \begin{itemize}
      \item \alert{Ensembles:} have only MD (content, access permissions, \ldots)\\
        \hspace*{5em} {\tt \alert{mc}://\param{rg}/\param{collab}/\param{proj}/...} 
      \item Configurations: MD (related ensemble, provenance info)\\
        {\bf and} actual data\\
        \hspace*{5em} {\tt lfn://\param{rg}/\param{collab}/\param{proj}/...} 
      \end{itemize}

      \column{0.35\textwidth}
      \includegraphics[height=17mm]{\figs/md-org}
    \end{columns}

    \begin{center}
      {\footnotesize
        \begin{tabular}{lccccc}
          ID & entity           & relation   & content & data storage & access control \\\hline
          \tt lfn & config      & mc         & yes     & yes  &  no            \\
          & $\downarrow$&            &         &      &  $\uparrow$    \\
          \tt \alert{mc}  & \alert{ensemble}    & ---        & yes     &  no  &  yes      \\
          & $\uparrow\uparrow\uparrow$ \\
          $*)$  & \color{blue}{publication} & set of mc  & yes     &  no            & no 
        \end{tabular}
      }
    \\[5mm]
    $*)$ ILDG 1.0 has no official registration of IDs or publication metadata yet!
    \end{center}
\end{frame}
%------------------------------------------------
\begin{frame}{Introducing DOI}

Example of DOI: \newline
https://doi.org/10.22323/1.430.0203 

\vspace{1 cm}
\begin{itemize}
    \item A digital object identifier (DOI) is a persistent identifier or handle used to uniquely identify various objects, standardized by the International Organization for Standardization (ISO).

\item  DOIs are widely used to identify academic, professional, and government information, such as journal articles, research reports, data sets, and official publications. 
    
\end{itemize}

\url{https://en.wikipedia.org/wiki/Digital_object_identifier}

\end{frame}

%%%%%%%%%%%%%%%%%%%%%%%%%%%%
\begin{frame}{DOI and Data Publishing}
  \begin{block}{Data Publishing}
    \begin{itemize}
    \item Registration of persistent identifier (DOI)
    \item Metadata for registration (DataCite)
    \item Landing Page (hosting and automatic generation)
    \item Harvesting of metadata
    \end{itemize}
  \end{block}

  Exploratory setups by \href{https://www.jldg.org/DOI}{JLDG} and
  \href{https://www.osti.gov/dataexplorer/search/product-type:Dataset/semantic:Lattice QCD}{USQCD}
  \begin{itemize}
  \item using national registration authorities (JaLC, OSTI)
  \item workflow and metadata for registration and generation of landing pages
  \end{itemize}

  Possible directions in ILDG 2.0
  \begin{itemize}
  \item establish workflow for registration, generation and hosting of landing pages
    (e.g. \href{https://zenodo.org/communities/ildg}{Zenodo})
  \item extended metadata support
  \item dedicated metadata harvesting (e.g. by INSPIRE)
  \item common registration authority
  \end{itemize}
  
\end{frame}
    
%------------------------------------------------
\begin{frame}{What does ``accessible'' mean?}
  \begin{alertblock}{Accessible}
    \begin{itemize}
      \item[A1~~] (M)D retrievable by ID using standardized protocols 
      \item[A1.1] protocol is open, free, and universally implementable 
      \item[A1.2] protocol allows authentication/authorization procedure where necessary 
      \item[A2~~] MD accessible even if data is no longer available
    \end{itemize}
  \end{alertblock}

  \begin{itemize}
  \item A1 can be achieved e.g. by a File Catalog: ID $\mapsto$ storage location(s)
  \item Accessible does not imply (unrealistic) public access without authentication
  \item MD is precious even without the associated data 
  \end{itemize}

  \vfill

\end{frame}
%------------------------------------------------
\begin{frame}{How does ILDG address ``accessible''?}
  \begin{center}
    \includegraphics[height=40mm]{\figs/arch-ildg-db}
  \end{center}
  \begin{itemize}
  \item all metadata is \alert{publicly} accessible (from MDC)
  \item well-defined community-wide metadata \alert{schema}
  \item metadata available in a standard \alert{markup} language 
  \item standardized protocols and API of \alert{services} for access to data and metadata
  \end{itemize}
\end{frame}
%------------------------------------------------
\begin{frame}{What does ``interoperable'' mean?}
  \begin{alertblock}{Interoperable}
    \begin{itemize}
    \item[I1] (M)D use a formal, accessible, shared, and broadly applicable 
      language % for knowledge representation
    \item[I2] (M)D use vocabularies that follow FAIR principles
    \item[I3] (M)D include qualified references to other (M)D
    \end{itemize}
  \end{alertblock}

  \begin{itemize}
  \item ability of data (or tools) from non-cooperating resources\\
    to integrate (or work together) with minimal effort
    
  \end{itemize}
\end{frame}
%------------------------------------------------
\begin{frame}{How does ILDG address ``interoperable''?}

  \begin{block}{Common standards for}
    \begin{itemize}
    \item Metadata schema
    \item Data format
    \item API and URL for web services of regional grids
    \end{itemize}
  \end{block}

  New directions:
  \begin{itemize}
  \item Extend ILDG format to include support for \alert{HDF5}
    \begin{itemize}
    \item definition of ILDG packing rules
    \item convenient tools for packing and conversion
    \end{itemize}
  \item Token-based authentication
  \item REST API
  \end{itemize}
  
\end{frame}
%------------------------------------------------
\begin{frame}{What does ``reusable'' mean?}
  \begin{alertblock}{Reusable}
    \begin{itemize}
    \item[R1~~] (M)D richly described with plurality of accurate and relevant attributes
    \item[R1.1] (M)D released with clear and accessible data usage license
    \item[R1.2] (M)D associated with detailed provenance
    \item[R1.3] (M)D meet domain-relevant community standards
    \end{itemize}
  \end{alertblock}

  \begin{itemize}
  \item reference to a paper may not be sufficient
  \item good scientific practice $\leftrightarrow$ FAIR
  \item also related to verifiable invariance of results
    \hfill {\small \href{https://inspirehep.net/literature/2601251}  {  (see presentation by Ed Bennett)}   }
    \begin{itemize}
    \item reproducibility: same data + same analysis
    \item replicability: new data + same analysis
    \item robustness: same data + new analysis
    \end{itemize}
  \end{itemize}

  \vfill

\end{frame}
%------------------------------------------------
\begin{frame}{How does ILDG address ``reusable''?}
  \begin{columns}[c] 
    \column{0.48\textwidth}
    {\bf Ensemble MD}
    \begin{itemize}
    \item Physics
    \item Algorithm
    \item Management
    \end{itemize}

    \vspace*{5mm}
    {\bf Config MD}
    \begin{itemize}
    \item Markov step
    \item Implementation (machine, code)
    \item Management (creator, date, checksum)
    \end{itemize}

    \vspace*{5mm}
    But \alert{no} license aspects! (cf. R1.1)

    \column{0.5\textwidth}
    \hspace*{-5mm}\includegraphics[height=65mm]{\figs/ensemble-tree}
  \end{columns}
\end{frame}

\end{document}


